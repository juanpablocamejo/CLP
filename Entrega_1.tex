\documentclass[a4paper,10pt]{article} % a4paper es obvio y 10pt es el tamaño
                                      % de la tipografía básica

\usepackage[utf8]{inputenc} % Permite escribir acentos (leer este archivo en 
                            % UTF8!)

\usepackage[T1]{fontenc}    % Permite que el texto que se produce se pueda
                            % copiar (incluídos los símbolos)

\usepackage[spanish]{babel} % Hace que las palabras claves (como la fecha)
                            % salgan en español

\usepackage{fullpage}       % Pone márgenes más chicos que los estándares

\setlength\parindent{0em}   % Elimina la identación

\usepackage{amsmath,amssymb}        % Provee el entorno align*, entre otros

\usepackage{url}            % No necesario en general. Es para poner links 
                            % (como el que puse a overleaf)

\usepackage{proof}          % Este paquete da el comando \infer. 

\usepackage{logicproof}          % Sirve para escribir pruebas en estilo Fitch como lo hace Huth y Ryan

% Más documentación:
% http://www.actual.world/resources/tex/doc/Proofs.pdf

\title{Entrega 1}
\author{Juan Pablo Camejo (19803)}
\date{\today}  % En vez de \today pueden poner el texto que quieran (como la 
               % fecha de ayer por ejemplo). Si ponen \today, queda la fecha
               % de compilación

\begin{document}


\maketitle   % Pone el título autor y fecha definidos antes

\subsection*{Ejercicio}
Definir inductivamente el conjunto de variables ligadas $bv$ en PCF.

\subsection*{Solución}

$$
\begin{aligned}
    bv(x) &= \emptyset\\
    bv(\lambda x.t) &= \{x\} \cup bv(t) \\
    bv(tu) &= bv(t)\cup bv(u)\\
    bv(n) &= \emptyset\\
    bv(t\odot u) &= bv(t)\cup bv(u)\\
    bv(\text{ifz } t \text{ then } u \text{ else } v) &= bv(t)\cup bv(u) \cup bv(v)\\
    bv(\mu x.t) &= \{x\} \cup bv(t)\\
    bv(\text{let } x = t \text{ in } u) &= \{x\} \cup bv(t) \cup bv(u)\\
\end{aligned}
$$
\end{document}