\documentclass[a4paper,10pt]{article} % a4paper es obvio y 10pt es el tamaño

\usepackage[utf8]{inputenc} % Permite escribir acentos (leer este archivo en 
                            % UTF8!)

\usepackage[T1]{fontenc}    % Permite que el texto que se produce se pueda
                            % copiar (incluídos los símbolos)

\usepackage[spanish]{babel} % Hace que las palabras claves (como la fecha)
                            % salgan en español


\usepackage{fullpage}       % Pone márgenes más chicos que los estándares

\setlength\parindent{0em}   % Elimina la identación

\usepackage{amsmath,amssymb}        % Provee el entorno align*, entre otros

\usepackage{url}            % No necesario en general. Es para poner links 
                            % (como el que puse a overleaf)

\usepackage{proof}          % Este paquete da el comando \infer. 

\usepackage{logicproof}          % Sirve para escribir pruebas en estilo Fitch como lo hace Huth y Ryan
\usepackage{xcolor}
\usepackage{ulem}
\usepackage{proof}

% Más documentación:
% http://www.actual.world/resources/tex/doc/Proofs.pdf

\title{Entrega 2}
\author{Juan Pablo Camejo (19803)}
\date{\today}  % En vez de \today pueden poner el texto que quieran (como la 
               % fecha de ayer por ejemplo). Si ponen \today, queda la fecha
               % de compilación

\begin{document}


\maketitle   % Pone el título autor y fecha definidos antes

\section*{Ejercicio}
Tipar el siguiente termino: $(\lambda x:A.\lambda y:B .x+yx)$\\


\subsection*{Solución}
$$
\infer[\Rightarrow_i]{\vdash (\lambda x:nat .\lambda y:B .x+yx) :nat \Rightarrow nat}{
\infer[\Rightarrow_i]{  x:nat \vdash (\lambda y:nat\Rightarrow nat.x + y x) : nat \Rightarrow nat \Rightarrow nat }{
\infer[\otimes]{\Gamma = x:nat,y:nat\Rightarrow nat \vdash  (x + y x) : nat}{
    \infer[ax_v]{\Gamma\vdash x:nat}{} & 
    \infer[\Rightarrow_e]{\Gamma \vdash (yx):nat}{ \infer[ax_v]{\Gamma \vdash y :nat \Rightarrow nat}{} &  \infer[ax_v]{\Gamma \vdash x:nat}{} }
                     }
                      }}
$$
\end{document}