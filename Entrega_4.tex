\documentclass[a4paper,10pt]{article} % a4paper es obvio y 10pt es el tamaño

\usepackage[utf8]{inputenc} % Permite escribir acentos (leer este archivo en 
                            % UTF8!)

\usepackage[T1]{fontenc}    % Permite que el texto que se produce se pueda
                            % copiar (incluídos los símbolos)

\usepackage[spanish]{babel} % Hace que las palabras claves (como la fecha)
                            % salgan en español


\usepackage{fullpage}       % Pone márgenes más chicos que los estándares

\setlength\parindent{0em}   % Elimina la identación

\usepackage{amsmath,amssymb}        % Provee el entorno align*, entre otros

\usepackage{url}            % No necesario en general. Es para poner links 
                            % (como el que puse a overleaf)

\usepackage{proof}          % Este paquete da el comando \infer. 

\usepackage{logicproof}          % Sirve para escribir pruebas en estilo Fitch como lo hace Huth y Ryan
\usepackage{xcolor}
\usepackage{ulem}
\usepackage{proof}
\newenvironment{blue}{\Color{blue}}{\Color{black}}
\newcommand{\En}[0]{\Rightarrow}



% Más documentación:
% http://www.actual.world/resources/tex/doc/Proofs.pdf

\title{Entrega 4}
\author{Juan Pablo Camejo (19803)}
\date{\today}  % En vez de \today pueden poner el texto que quieran (como la 
               % fecha de ayer por ejemplo). Si ponen \today, queda la fecha
               % de compilación

\begin{document}


\maketitle   % Pone el título autor y fecha definidos antes

\section*{Ejercicio}
Tipar los siguientes terminos aplicando Hindley y Robinson: 
\subsubsection*{1) $\lambda x.\lambda y .xy$}
\subsubsection*{2) $\lambda x.\lambda y .xy+yx$}


\section*{Solución}



\subsubsection*{1) $\lambda x.\lambda y .xy$}
$$
\infer[]{ y:Y, x:X \vdash \lambda x.\lambda y.xy \leadsto  X \En Y \En Z, \tau = \{X=Y\En Z\} }{
\infer[]{ y:Y, x:X \vdash \lambda y.xy \leadsto Y \En Z,\{X=Y\En Z\} }{
\infer[]{ y:Y, x:X \vdash xy \leadsto Z,\{X=Y\En Z\} }{
    \infer[]{ y:Y, x:X \vdash x \leadsto X,\emptyset}{} &  \infer[]{x:X, y:Y \vdash y \leadsto Y,\emptyset}{}}
}}
$$
Al aplicar Robinson, no entra en ninguna regla, por lo que el conjunto $\tau$ queda igual.\\ 
$$\theta=\{Y\En Z/X\}$$
$$ 
    \text{El tipo es }\theta(X \En Y \En Z) = (Y \En Z) \En Y \En Z
$$
\subsubsection*{2) $\lambda x.\lambda y .xy+yx$}
$$
\infer[]{
\infer[]{
    \infer[]{x:X,y:Y\vdash \lambda x.\lambda y.xy+yx \leadsto X \En Y \En nat,\tau}{
        x:X,y:Y\vdash \lambda y.xy+yx \leadsto Y \En nat,\tau}}{
        x:X,y:Y\vdash xy+yx \leadsto nat, \tau =\{X=Y \En W,Y=X \En Z,W=nat,Z=nat\}}}
{
\infer[]{ x:X,y:Y\vdash xy \leadsto W,\{X=Y \En W\}}{
    \infer[]{y:Y,x:X\vdash x\leadsto X,\emptyset}{} & \infer[]{x:X,y:Y\vdash y\leadsto Y,\emptyset}{}}
&
\infer[]{ x:X,y:Y\vdash yx \leadsto Z,\{Y=X \En Z\}}{
\infer[]{x:X,y:Y\vdash y\leadsto Y,\emptyset}{} & \infer[]{y:Y,x:X\vdash x\leadsto X,\emptyset}{}}
}
$$
Aplico Robinson en el sistema de ecuaciones:
\begin{equation*}
    \begin{cases}
    W=nat\\
    Z=nat\\
    X=Y \En W \\
    Y=X \En Z \\
  \end{cases}
  \to
  \begin{cases}
    W=nat\\
    Z=nat\\
    X=Y \En nat \\
    Y=X \En Z \\
  \end{cases}
  \to
  \begin{cases}
    W=nat\\
    Z=nat\\
    X=Y \En nat \\
    Y=X \En nat \\
  \end{cases}
  \to
  \begin{cases}
    W=nat\\
    Z=nat\\
    X=Y \En nat \\
    \color{red}Y=(Y \En nat) \En nat \\
  \end{cases}
\end{equation*}
Falla luego del tercer paso, por la quinta regla, que dice:\\
Si una ecuación tiene forma $A = X, X = A$ y $X$ aparece en $A$, pero $A \neq X$,
responder error.
\end{document}