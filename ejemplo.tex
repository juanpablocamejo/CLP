\documentclass[a4paper,10pt]{article} % a4paper es obvio y 10pt es el tamaño
                                      % de la tipografía básica

\usepackage[utf8]{inputenc} % Permite escribir acentos (leer este archivo en 
                            % UTF8!)

\usepackage[T1]{fontenc}    % Permite que el texto que se produce se pueda
                            % copiar (incluídos los símbolos)

\usepackage[spanish]{babel} % Hace que las palabras claves (como la fecha)
                            % salgan en español

\usepackage{fullpage}       % Pone márgenes más chicos que los estándares

\setlength\parindent{0em}   % Elimina la identación

\usepackage{amsmath,amssymb}        % Provee el entorno align*, entre otros

\usepackage{url}            % No necesario en general. Es para poner links 
                            % (como el que puse a overleaf)

\usepackage{proof}          % Este paquete da el comando \infer. 

\usepackage{logicproof}          % Sirve para escribir pruebas en estilo Fitch como lo hace Huth y Ryan

% Más documentación:
% http://www.actual.world/resources/tex/doc/Proofs.pdf

\title{Entrega 2}
\author{Juan Pablo Camejo (19803)}
\date{\today}  % En vez de \today pueden poner el texto que quieran (como la 
               % fecha de ayer por ejemplo). Si ponen \today, queda la fecha
               % de compilación

\begin{document}


\maketitle   % Pone el título autor y fecha definidos antes

\section{Ejercicio 1}
Si $\phi$ es $\psi\lor\theta$ entonces, por hipótesis inductiva, 
$\verb|v|(\psi)=\verb|v|^\prime(\psi)$ y $\verb|v|(\theta)=\verb|v|^\prime(\theta)$.
Queremos probar que la igualdad se mantiene para $\lor$.

$$
\begin{aligned}
\verb|v|(\phi) = \verb|v|(\psi\lor\theta) 
            &= \left\{
                \begin{aligned}
                        & T \text{ si } \verb|v|(\psi)=T \text{ o } \verb|v|(\theta)=T\\
                        & F \text{ si } \verb|v|(\psi)=F \text{ y } \verb|v|(\theta)=F \\
                \end{aligned}
                \right. & \text{(def.)}\\
            &= \left\{
                \begin{aligned}
                        & T \text{ si } \verb|v|^\prime(\psi)=T \text{ o } \verb|v|^\prime(\theta)=T\\
                        & F \text{ si } \verb|v|^\prime(\psi)=F \text{ y } \verb|v|^\prime(\theta)=F \\
                \end{aligned}
                \right. & \text{(H.I.)}\\
            &= \verb|v|^\prime(\phi)
\end{aligned}
$$

Si $\phi$ es $\psi\to\theta$ entonces, por hipótesis inductiva, 
$\verb|v|(\psi)=\verb|v|^\prime(\psi)$ y $\verb|v|(\theta)=\verb|v|^\prime(\theta)$.
Queremos probar que la igualdad se mantiene para $\to$.$$
\begin{aligned}
\verb|v|(\phi) = \verb|v|(\psi\to\theta) 
            &= \left\{
                \begin{aligned}
                        & F \text{ si } \verb|v|(\psi)=T \text{ y } \verb|v|(\theta)=F\\
                        & T \text{ si } \verb|v|(\psi)=F \text{ o } \verb|v|(\theta)=T \\
                \end{aligned}
                \right. & \text{(def.)}\\
            &= \left\{
                \begin{aligned}
                    & F \text{ si } \verb|v|^\prime(\psi)=T \text{ y } \verb|v|^\prime(\theta)=F\\
                    & T \text{ si } \verb|v|^\prime(\psi)=F \text{ o } \verb|v|^\prime(\theta)=T \\
                \end{aligned}
                \right. & \text{(H.I.)}\\
            &= \verb|v|^\prime(\phi)
\end{aligned}
$$

\section{Ejercicio 2}

Deseamos probar que una fórmula $\phi$ es una tautología si y sólo si $\neg\phi$ es insatisfacible.
\\\\
$\vDash\phi$ $\to$ $\nvDash\neg\phi$ ya fué probado en clase.
\\\\
Sólo nos queda probar que $\nvDash\neg\phi$ $\to$ $\vDash\phi$: 
$$
\begin{aligned}
&\text{Como} \neg\phi \text{ es no satisfacible} &\nexists\verb|v| :. \verb|v|(\neg\phi)=T \\
&\text{entonces} &\forall\verb|v| \text{ }  \verb|v|(\neg\phi)=F\\
&\text{por lo tanto} &\forall\verb|v| \text{ } \verb|v|(\phi) = T\\
&\text{quiere decir que} &\phi \text{ es una tautología.}
\end{aligned}
$$

\end{document}