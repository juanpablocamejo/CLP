\documentclass[a4paper,10pt]{article} % a4paper es obvio y 10pt es el tamaño

\usepackage[utf8]{inputenc} % Permite escribir acentos (leer este archivo en 
                            % UTF8!)

\usepackage[T1]{fontenc}    % Permite que el texto que se produce se pueda
                            % copiar (incluídos los símbolos)

\usepackage[spanish]{babel} % Hace que las palabras claves (como la fecha)
                            % salgan en español


\usepackage{fullpage}       % Pone márgenes más chicos que los estándares

\setlength\parindent{0em}   % Elimina la identación

\usepackage{amsmath,amssymb}        % Provee el entorno align*, entre otros

\usepackage{url}            % No necesario en general. Es para poner links 
                            % (como el que puse a overleaf)

\usepackage{proof}          % Este paquete da el comando \infer. 

\usepackage{logicproof}          % Sirve para escribir pruebas en estilo Fitch como lo hace Huth y Ryan
\usepackage{xcolor}
\usepackage{ulem}
\usepackage{proof}
\newenvironment{blue}{\Color{blue}}{\Color{black}}
\newcommand{\To}[0]{\Rightarrow}
\newcommand{\vd}[0]{\vdash}
\newcommand\fun[2]{\ensuremath{\lambda {#1}{.}{#2}}}



% Más documentación:
% http://www.actual.world/resources/tex/doc/Proofs.pdf

\title{Entrega 5}
\author{Juan Pablo Camejo (19803)}
\date{\today}  % En vez de \today pueden poner el texto que quieran (como la 
               % fecha de ayer por ejemplo). Si ponen \today, queda la fecha
               % de compilación

\begin{document}


\maketitle   % Pone el título autor y fecha definidos antes

\section*{Ejercicio}
Tipar el siguiente termino, usando tipos polimórficos: 
\subsubsection*{ $(\fun f{ff})(\fun xx)$}


\section*{Solución}

$$
\infer[\To_e]{(\vd \fun f{ff})(\fun xx) :[B]}{
      \infer[\To_i]{ \vd\fun f{ff} :[(A \To A) \To B] }{
          \infer[\To_e]{f:[A \To A] \vd ff : [B]}{
               f:[A \To A] \vd \color{red}f:[(A \To A)\To B]
             & f:[A \To A] \vd \color{red}f:[A \To A]
          }
      } 
    & \infer[\To_i]{\vd \fun xx : [A \To A]}{ 
          \infer[ax_v]{x:[A]\vd x:[A]}{} 
        & \infer[ax_v]{x:[A]\vd x:[A]}{}
    }
}
$$

Al aplicar la regla de $\To_e$ se llega a una contradicción entre los tipos marcados en rojo.\\
(En este punto no se puede agregar cuantificadores porque la regla de $\To_e$ no lo permite)\\
Por lo tanto, el termino $(\fun f{ff})(\fun xx)$ no se puede tipar.
\end{document}